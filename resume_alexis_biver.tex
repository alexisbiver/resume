\documentclass{resume} % Use the custom resume.cls style
\usepackage[left=0.4 in,top=0in,right=0.4 in,bottom=0.4in]{geometry} % Document margins
\newcommand{\tab}[1]{\hspace {.2667\textwidth}\rlap{#1}} 
\newcommand{\itab}[1]{\hspace{0em}\rlap{#1}}

\begin{document}

% Title
\begin{center}
    {\Huge \textbf{Alexis Biver}} \hspace{0.3cm} biver.alexis@proton.me \hspace{2.7cm} Metz, Grand Est, France \\
\end{center}

% \vspace{0.2cm}
\begin{center}
\textbf{Machine Learning Engineer} with 6 years of experience in Deep Learning and Data Science. \\
I have innovated in domains such as Robotics, Voice Assistants and Data, from Proof of Concept to Deployment. \\
Curious and eager to learn, I love to apply the latest techniques to develop new solutions for any domain.
\end{center}


% Section: Core Skills
% \vspace{0.2cm}
\begin{rSection}{Core Skills}
\begin{itemize}
    \item \textbf{Tools \& Deployment:} Python, PyTorch, Scikit-learn, Docker, CI/CD, FASTApi, ONNX, C++, SQL, ROS
    \item \textbf{Machine Learning:} NLP, Time Series Forecasting, Edge-device Optimization, Diffusion Models
    \item \textbf{Data:} Data Analysis, ETL, Feature Engineering, Data Visualization (Matplotlib, PowerBI)
    \item \textbf{Languages:} French (Native), English (Professional Working Proficiency)
\end{itemize}
\end{rSection}

% Section: Professional Experience
% \vspace{0.2cm}
\begin{rSection}{Professional Experience}
\textbf{Machine Learning Engineer} \hfill \textit{Inria, Team Larsen/HuCeBot} \\
Nancy, Grand Est, France \hfill Jan 2023 -- Dec 2024 (2 years)
\begin{itemize}
    \item Developed an open-source \href{https://github.com/hucebot/prescyent}{Python library} for trajectory prediction in robotics
        \item Performed benchmarks and implementations of the State Of The Art solutions.
        \item Preprocessed and analyzed multimodal sensor data (LIDAR, IMU, Cameras).
    \item Supervised an internship leading to a software allowing simple controls of a humanoid robot by voice
\end{itemize}

\vspace{0.1cm}

\textbf{Teaching Associate} \hfill \textit{IUT Nancy-Charlemagne} \\
Nancy, Grand Est, France \hfill March 2024 -- June 2024  (1 semester)
\begin{itemize}
    \item Delivered lectures and practical sessions on "Introduction to Network Services" for first-year IT students.
    \item Conducted Java-based exercises and evaluations to assess students' understanding.
\end{itemize}

\vspace{0.1cm}

\textbf{Machine Learning Engineer} \hfill \textit{Vivoka, R\&D} \\
Metz, France \hfill Sept 2019 -- Jan 2023 (3 years, 4 months)
\begin{itemize}
    \item Designed and optimized ML models for edge devices using PyTorch and ONNX, with Python and C++.
    \item Developed Python APIs to serve ML solutions and ensured production readiness with Docker.
    \item Conducted research, benchmarking, and technical popularization of scientific advancements.
    \item Enhanced data collection and processing to train robust NLP models for voice assistants.
\end{itemize}

\vspace{0.1cm}

\textbf{Data Analyst Developer (Internship)} \hfill \textit{PwC Luxembourg, Data Services} \\
Luxembourg \hfill March 2019 -- Aug 2019 (6 months)
\begin{itemize}
    \item Executed reporting and data visualization using ETL tools and Power BI.
    \item Updated data-mining solutions in C\# and SQL to improve data quality and efficiency.
    \item Researched and implemented a natural language understanding (NLU) approach for data augmentation.
\end{itemize}

\end{rSection}

% Section: Education
% \vspace{0.2cm}
\begin{rSection}{Education}

\textbf{Master's in Natural Language Processing} \\
\textit{Institut des sciences du Digital, Management et Cognition (IDMC)} \hfill 2017--2019

\textbf{Bachelor's in Mathematics and Computer Science} \\
\textit{Université de Lorraine} \hfill 2014--2017 \\
\end{rSection}
\end{document}
