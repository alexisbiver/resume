\documentclass{resume} % Use the custom resume.cls style
\usepackage[left=0.4 in,top=0in,right=0.4 in,bottom=0.4in]{geometry} % Document margins
\newcommand{\tab}[1]{\hspace {.2667\textwidth}\rlap{#1}} 
\newcommand{\itab}[1]{\hspace{0em}\rlap{#1}}

\begin{document}

% Title
\begin{center}
    {\Huge \textbf{Alexis Biver}} \hspace{0.3cm} biver.alexis@proton.me \hspace{2.7cm} Metz, Grand Est, France \\
\end{center}

% \vspace{0.2cm}
\begin{center}
\textbf{Machine Learning Engineer} with 6 years of experience in Deep Learning and Data Science. \\
I have innovated in domains such as Robotics, Voice Assistants and Data, from Proof of Concept to Deployment. \\
Curious and eager to learn, I love to apply the latest techniques to develop new solutions for any domain.
\end{center}


% Section: Core Skills
% \vspace{0.2cm}
\begin{rSection}{Core Skills}
\begin{itemize}
    \item \textbf{Tools \& Deployment:} Python, PyTorch, Scikit-learn, Seaborn, C++, ROS, SQL,  Docker, CI/CD, Git
    \item \textbf{Machine Learning:} NLP (Embedding, Transformer, LLM), Finetuning, Regression Models, Diffusion Models
    \item \textbf{Data:} Data Analysis, Statistical Modeling, ETL, Feature Engineering, Data Visualization
    \item \textbf{Languages:} French (Native), English (Professional Working Proficiency)
\end{itemize}
\end{rSection}

% Section: Professional Experience
% \vspace{0.2cm}
\begin{rSection}{Professional Experience}
\textbf{Machine Learning Engineer} \hfill \textit{Inria, Team Larsen/HuCeBot} \\
Nancy, Grand Est, France \hfill Jan 2023 -- Dec 2024 (2 years)
\begin{itemize}
    \item Developed an open-source \href{https://github.com/hucebot/prescyent}{Python library} for trajectory prediction in robotics, leveraging PyTorch and Scipy.
    \item Performed benchmarks and implementations of the State Of The Art solutions (MLPs, Transformers, Seq2Seq...)
    \item Preprocessed, analyzed and tuned multimodal sensor data (LIDAR, IMU, Cameras) to train predictive models.
    \item Supervised an internship leading to Python software allowing controls of a humanoid robot by voice using ROS.
\end{itemize}

\vspace{0.1cm}

\textbf{Teaching Associate} \hfill \textit{IUT Nancy-Charlemagne} \\
Nancy, Grand Est, France \hfill March 2024 -- June 2024  (1 semester)
\begin{itemize}
    \item Delivered lectures and practical sessions on "Introduction to Network Services" for first-year IT students.
    \item Conducted Java-based exercises and evaluations to assess students' understanding.
\end{itemize}

\vspace{0.1cm}

\textbf{Machine Learning Engineer} \hfill \textit{Vivoka, R\&D} \\
Metz, France \hfill Sept 2019 -- Jan 2023 (3 years, 4 months)
\begin{itemize}
    \item Finetuned LLMs for classification tasks on edge devices using PyTorch and ONNX, with Python and C++.
    \item Developed Python APIs (SQL, FastAPI) to serve ML solutions and ensured production readiness with Docker.
    \item Enhanced data collection and processing to train robust NLP models for voice assistants (NLU and ASR).
    \item Conducted research, benchmarking, and technical popularization of scientific papers.
\end{itemize}

\vspace{0.1cm}

\textbf{Data Analyst Developer (Internship)} \hfill \textit{PwC Luxembourg, Data Services} \\
Luxembourg \hfill March 2019 -- Aug 2019 (6 months)
\begin{itemize}
    \item Executed reporting and data visualization using ETL tools (SSIS and SQL) and Power BI.
    \item Updated data-mining solutions in C\# and SQL to improve data aquisition processes.
    \item Researched and implemented a natural language understanding (NLU) approach for data augmentation.
\end{itemize}

\end{rSection}

% Section: Education
% \vspace{0.2cm}
\begin{rSection}{Education}

\textbf{Master's in Natural Language Processing} \\
\textit{Institut des sciences du Digital, Management et Cognition (IDMC)} \hfill 2017--2019

\textbf{Bachelor's in Mathematics and Computer Science} \\
\textit{Université de Lorraine} \hfill 2014--2017 \\
\end{rSection}
\end{document}
