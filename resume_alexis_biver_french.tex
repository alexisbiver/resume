\documentclass{resume} % Utilisation du style personnalisé resume.cls
\usepackage[left=0.4 in,top=0in,right=0.4 in,bottom=0.4in]{geometry} % Marges du document
\newcommand{\tab}[1]{\hspace {.2667\textwidth}\rlap{#1}} 
\newcommand{\itab}[1]{\hspace{0em}\rlap{#1}}

\begin{document}

% Titre
\begin{center}
    {\Huge \textbf{Alexis Biver}} \hspace{5cm} Metz, Grand Est, France \\
    \href{mailto:biver.alexis@proton.me}{biver.alexis@proton.me}   \hspace{1cm}     \href{https://github.com/alexisbiver}{www.github.com/alexisbiver}  \hspace{1cm} \href{https://linkedin.com/in/biver-alexis}{linkedin.com/in/biver-alexis}
\end{center}

\vspace{0.5cm}
\begin{center}
\textbf{Ingénieur Machine Learning} avec plus de 5 ans d'expérience en Recherche et Développement dans des domaines comme la Robotique et Traitement Automatique des Langues (TAL).
Curieux et avide d'apprendre, j'aime appliquer les techniques les plus récentes pour développer des solutions innovantes dans divers domaines.
\end{center}

% Section : Compétences Clés
\begin{rSection}{Compétences Clés}
\begin{itemize}
    \item \textbf{Programmation \& Outils :} Python, Torch, Docker, FASTApi, ONNX, C++, TS, MySQL, ROS
    \item \textbf{Apprentissage Automatique :} Prédiction de mouvements, NLP, Optimisation pour appareils embarqués
    \item \textbf{Langues :} Français (langue maternelle), Anglais (niveau professionnel)
\end{itemize}
\end{rSection}

% Section : Expériences Professionnelles
\begin{rSection}{Expériences Professionnelles}
\textbf{Ingénieur Machine Learning} \hfill \textit{Inria, Équipe Larsen/HuCeBot} \\
Nancy, Grand Est, France \hfill Jan 2023 -- Déc 2024 (2 ans)
\begin{itemize}
    \item Développement d'une \href{https://github.com/hucebot/prescyent}{librairie Python open-source} pour la prédiction de trajectoires en robotique.
    \item Réalisation de benchmarks et implémentation des solutions SOTA pour la librairie.
    \item Encadrement d’un stage ayant conduit à un logiciel permettant le contrôle vocal simple d’un robot humanoïde.
\end{itemize}

\vspace{0.2cm}

\textbf{Chargé d'enseignement} \hfill \textit{IUT Nancy-Charlemagne} \\
Nancy, Grand Est, France \hfill Mars 2024 -- Juin 2024 (1 semestre)
\begin{itemize}
    \item Cours et TP "Introduction aux Services Réseaux" pour des étudiants de première année en informatique.
    \item Evaluations des connaissances via des exercices papiers et en Java.
\end{itemize}

\vspace{0.2cm}

\textbf{Ingénieur Machine Learning} \hfill \textit{Vivoka, R\&D} \\
Metz, France \hfill Sept 2019 -- Jan 2023 (3 ans, 4 mois)
\begin{itemize}
    \item Conception et optimisation des modèles ML pour appareils embarqués avec PyTorch et ONNX.
    \item Développement des APIs Python pour déployer des solutions ML prêtes pour la production.
    \item Recherche, benchmarks, présentations et vulgarisation sur les dernières avancées scientifiques.
    \item Collecte et traitement des données pour entraîner des modèles NLP robustes.
\end{itemize}

\vspace{0.2cm}

\textbf{Développeur Analyste de Données (Stage)} \hfill \textit{PwC Luxembourg, Data Services} \\
Luxembourg \hfill Mars 2019 -- Août 2019 (6 mois)
\begin{itemize}
    \item Rapporting et des visualisations de données à l’aide d’outils ETL et Power BI.
    \item Mise à jour des solutions de data-mining en C\#.
    \item Recherche et implémentation d'une approche NLU pour l’augmentation de données.
\end{itemize}

\end{rSection}

% Section : Formation
\begin{rSection}{Formation}

\textbf{Master en Traitement Automatique des Langues (TAL)} \\
\textit{Institut des sciences du Digital, Management et Cognition (IDMC)} \hfill 2017--2019

\textbf{Licence en Mathématiques et Informatique} \\
\textit{Université de Lorraine} \hfill 2014--2017 \\
\end{rSection}
\end{document}
